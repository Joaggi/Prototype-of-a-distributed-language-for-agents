\documentclass{article}
\usepackage[spanish]{babel}
\usepackage{graphicx}
\usepackage[utf8]{inputenc}
\usepackage{listings}
\usepackage{hyperref} 


\begin{document}
\title{Prototipo de Lenguaje y Compilador para modelar redes AD-HOC usando agentes dispersos}
\author{Alegandro Gallego, Sergio Pérez, Jhon Ramírez}
\date{Aqui va la fecha}
\maketitle

\section{¿Qué es Pyro?}
Pyro significa \textbf{Py}thon \textbf{R}emote \textbf{O}bject, es una líbreria que permite construir aplicaciones, en las cuales los objetos pueden comunicarse entre ellos a través de una red, con un mínimo esfuerzo de programación. Con ayuda de esta líbreria Se puede hacer el llamado a los métodos de Python normalmente, con casi todos los posibles tipo de parámetros y valores de retorno, y Pyro se encarga de localizar el objeto correcto en el computador correcto para ejecutar el método.\\

Pyro está diseñado para ser muy fácil de usar, y generalmente no interponerse en el camino. También provee un conjunto de características poderosas que permiten construir aplicaciones distribuidas rápidamente y sin mucho esfuerzo. Pyro está desarrollada 100\% en Python puro y por lo tanto puede ser ejecutado en muchas plataformas y versiones de Python, incluyendo \textbf{Python 2.x}, \textbf{Python 3.x}, \textbf{IronPython}, \textbf{Jython 2.7+} and \textbf{Pypy}.\\

\begin{itemize}
\item Pyro es propiedad de Irmen de Jong \href{mailto:irmen@razorvine.net}{irmen@razorvine.net} - \url{http://www.razorvine.net}
\item El repositorio con el código fuente se encuentra en GitHub: \url{http://github.com/irmen/Pyro4}
\item La documentación de la librería se puede encontrar en: \url{http://pythonhosted.org/Pyro4/}
\item Pyro puede ser encontrado en \textbf{Pypi} como \href{https://pypi.python.org/pypi/Pyro4/}{Pyro4}
\end{itemize}

\section{Un poco de historia}
Pyro fue iniciado en 1998, hace más de diez años, cuando la tecnología de invocación de métodos remotos, tales como \textit{RMI} de Java y \textit{CORBA}, eran muy populares. El autor, quería algo así en Python, y como no había nada disponible, entonces decidió escribir el suyo propio. En el transcurso de los años lentamente se le fueron añadiendo nuevas características hasta la versión 3.10. En ese punto, era claro que el código base se había vuelto un poco viejo y no permitía la adición de nuevas características de una manera fácil, entonces, para inicios de 2010 se dió origen a Pyro4, escrito totalmente desde cero. Después de un par de versiones Pyro4 llegó a ser lo suficientemente estable para convertirse en la nueva versión "principal".\\

Pyro es el nombre del paquete de la versión antigua (\textit{3.x}) de Pyro. Pyro4 es el nombre del nuevo paquete, es decir, la versión actual. Su API y comportamiento es similar a Pyro 3.x, pero no es compatible hacia atras. Por ello, para evitar conflictos, la nueva versión de Pyro tiene un nombre diferente.

\section{¿Qué es el Name Server?}


\end{document}