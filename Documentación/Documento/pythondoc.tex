\documentclass{article}
\usepackage[spanish]{babel}
\usepackage{graphicx}
\usepackage[utf8]{inputenc}
\usepackage{listings}
\usepackage{hyperref} 
\usepackage{color}
\usepackage{listings}
\lstset{ %
language=C++,                % choose the language of the code
basicstyle=\footnotesize,       % the size of the fonts that are used for the code
numbers=left,                   % where to put the line-numbers
numberstyle=\footnotesize,      % the size of the fonts that are used for the line-numbers
stepnumber=1,                   % the step between two line-numbers. If it is 1 each line will be numbered
numbersep=5pt,                  % how far the line-numbers are from the code
backgroundcolor=\color{white},  % choose the background color. You must add \usepackage{color}
showspaces=false,               % show spaces adding particular underscores
showstringspaces=false,         % underline spaces within strings
showtabs=false,                 % show tabs within strings adding particular underscores
frame=single,           % adds a frame around the code
tabsize=2,          % sets default tabsize to 2 spaces
captionpos=b,           % sets the caption-position to bottom
breaklines=true,        % sets automatic line breaking
breakatwhitespace=false,    % sets if automatic breaks should only happen at whitespace
escapeinside={\%*}{*)}          % if you want to add a comment within your code
}

\begin{document}
\title{Documentación de código}
\author{Alegandro Gallego, Sergio Pérez, Jhon Ramírez}
\date{Aqui va la fecha}
\maketitle

\section{Documentación}
A continuación se presenta la descripción del comportamiento de cada una de las clases, así como el objetivo de cada uno de los métodos que las conforman, incluyendo sus parámetros de entrada y sus parametros de salida.

\subsection{Agente}
Representa un agente en el sistema. Está conformado por un componente de \textbf{Racionalidad}, y un componente de \textbf{Movilidad}, tiene la característica de dispersarse en la red, esto es, mover cada una de sus partes a otros nodos en la red, cada cierto intervalo de tiempo, determinado por una función de distribución de probabilidad.\\

\begin{lstlisting}
import Pyro4

class Agente(object):
    
    tipoMovilidad = ["constante","uniforme","exponencial"]    
    
    def __init__(self,nombre, movilidadId, racionalidadId, hostUri):
        self.hostUri = hostUri
        self.nombre = nombre
        self.movilidadId = movilidadId
        self.racionalidadId = racionalidadId

    def getMovilidadId(self):
        return self.movilidadId

    def getRacionalidadId(self):
        return self.racionalidadId
        
    def getNombre(self):
        return self.nombre
        
    def getType(self):
        return 'head'

    def getPyroId(self):
        return str(self._pyroId)

    def doIt(self):
        ##place some call to legs and arms
        racionalidadUri = Pyro4.Proxy(self.hostUri).resolve(self.racionalidadId)
        movilidadUri =  Pyro4.Proxy(self.hostUri).resolve(self.movilidadId)
        if (racionalidadUri == False or movilidadUri == False):
            return 'Algo esta perdido'
        racionalidad =x Pyro4.Proxy(racionalidadUri)
        movilidad = Pyro4.Proxy(movilidadUri)
        return [racionalidad.sayArms(), movilidad.sayLegs()]
\end{lstlisting}

\subsubsection*{Métodos}
A continuación se presentan los métodos de la clase Agente, se proporciona una descripción del objetivo que se persigue en cada método, así como la descripción de sus parámetros de entrada, y sus parámetros de salida que retorna una vez termine su ejecución.
\subsubsection{\textbf{init}(\textit{nombre}, \textit{movilidadId}, \textit{racionalidadId}, \textit{hostUri})}
Es el constructor de la clase, y a través de este método se permite instanciar objetos de esta clase.
\begin{lstlisting}
def __init__(self,nombre, movilidadId, racionalidadId, hostUri):
	self.hostUri = hostUri
	self.nombre = nombre
	self.movilidadId = movilidadId
	self.racionalidadId = racionalidadId
\end{lstlisting}
\subsubsection*{Parámetros}
\begin{itemize}
\item \textit{nombre}: Nombre con el que se identifica el Agente en la red.
\item \textit{movilidadId}: Identificador del objeto que representa la parte correspondiente a la capacidad del agente de moverse en la red.
\item \textit{racionalidadId}: Identificador del objeto que representa la parte correspondiente a la capacidad del agente de tomar decisiones.
\item \textit{hostUri}: Identificador de recurso uniforme del \textit{Host} donde está alojado el agente en un instante de tiempo específico.
\end{itemize}
\subsubsection{\textbf{getMovilidad}()}
Permite a los demás objetos en la red obtener el identificador del objeto de la movilidad del agente.
\begin{lstlisting}
def getMovilidadId(self):
	return self.movilidadId
\end{lstlisting}
\subsubsection*{Retorno}
Retorna el identificador del objeto que representa la parte correspondiente a la capacidad del agente de moverse en la red.
\subsubsection{\textbf{getRacionalidad}()}
Permite a los demás objetos en la red obtener el identificador del objeto de la racionalidad del agente.
\begin{lstlisting}
def getRacionalidadId(self):
	return self.racionalidadId
\end{lstlisting}
\subsubsection*{Retorno}
Retorna el identificador del objeto que representa la parte correspondiente a la capacidad del agente de tomar decisiones.
\subsubsection{nombre}
\begin{lstlisting}
codigo
\end{lstlisting}
\subsubsection*{Parámetros}
\subsubsection*{Retorno}
\end{document}